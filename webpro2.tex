\documentclass[uplatex,dvipdfmx]{jsarticle}
\usepackage[uplatex,deluxe]{otf} % UTF
\usepackage[noalphabet]{pxchfon} % must be after otf package
\usepackage{stix2} %欧文&数式フォント
\usepackage[fleqn,tbtags]{mathtools} % 数式関連 (w/ amsmath)
\usepackage{hira-stix} % ヒラギノフォント&STIX2 フォント代替定義(Warning回避)
\usepackage{booktabs}
\usepackage{listings}
\usepackage[utf8]{inputenc}
\usepackage{caption}
\captionsetup[lstlisting]{font={small,tt}}


\begin{document}
\title{webprograming期末レポート} %システム名 仕様書 という形式にする
\author{24G1089 武本龍}
\date{2025年1月7日}
\maketitle

\section{はじめに}
本レポートは


\section{プログラムの概要}
今回作成したプログラムの概要を説明する.



\subsection{プログラムの関係}
プログラムの中身を紹介する前に,作成したファイルの処理を説明する.
今回用いたファイルは以下の図の通りである.


 
自作関数を内包するファイルから「bigLetter.c」を説明のために使用する.今回用いたヘッダファイル「mul2.h」では関数や変数を宣言し,
main関数を内包しない「mul2.c」では「mul2.h」で宣言された関数の処理内容を記述し,
main関数を内包する「bigLetter.c」では実装された関数を呼び出し実行するという関係である.
「bigLetter.c」の部分は自作関数を内包するファイル全てに当てはまる.

\subsection{手順}
期末レポートで用いた掲示板のサーバーの立ち上げ方は以下の手順の通りである.

\subsubsection{サーバーの立ち上げ}
バックエンド側のサーバー(app8.js)を起動する手順は次の通りである.
\begin{enumerate}
    \item ターミナルを起動し、サーバー処理をするバックエンド側のJavaScriptファイルがあるディレクトリに移動する.(例:cd /Desktop/webpro/webpro\_06)
    \item 今回は app8.js がサーバー処理をするバックエンド側のJavaScriptなので、以下のコマンドを実行する.node app8.js
\end{enumerate}

\subsubsection{telnetによるサーバー接続}
別のターミナルを立ち上げて,telnetを使用してサーバーに接続する手順は次の通りである.
\begin{enumerate} 
    \item ターミナルを起動し、以下のコマンドを実行する.telnet localhost 8080
    \item 接続が成功したら、以下のコマンドを順に実行する.
    \begin{enumerate} 
        \item HTTPリクエストの開始行を入力する.GET /bbs HTTP/1.1
        \item 次に、ホスト名を指定する.Host: localhost
    \end{enumerate} 
    \item 最後に、空行(Enterキーを2回押す)を送信してリクエストを終了する. 
\end{enumerate}

